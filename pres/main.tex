%-=-=-=-=-=-=-=-=-=-=-=-=-=-=-=-=-=-=-=-=-=-=-=-=
%
%        LOADING DOCUMENT
%
%-=-=-=-=-=-=-=-=-=-=-=-=-=-=-=-=-=-=-=-=-=-=-=-=

\documentclass[newPxFont]{beamer}
\usefonttheme{serif}
\usetheme{sthlm}
%\usecolortheme{sthlmv42}

%-=-=-=-=-=-=-=-=-=-=-=-=-=-=-=-=-=-=-=-=-=-=-=-=
%        LOADING PACKAGES
%-=-=-=-=-=-=-=-=-=-=-=-=-=-=-=-=-=-=-=-=-=-=-=-=
\usepackage[T1]{fontenc}
\usepackage[utf8]{inputenc}
%\inputencoding{latin1}
%\usepackage[utf8x]{inputenc} 

\usepackage{chronology}


\renewcommand{\event}[3][e]{%
  \pgfmathsetlength\xstop{(#2-\theyearstart)*\unit}%
  \ifx #1e%
    \draw[fill=black,draw=none,opacity=0.5]%
      (\xstop, 0) circle (.2\unit)%
      node[opacity=1,rotate=45,right=.2\unit] {#3};%
  \else%
    \pgfmathsetlength\xstart{(#1-\theyearstart)*\unit}%
    \draw[fill=black,draw=none,opacity=0.5,rounded corners=.1\unit]%
      (\xstart,-.1\unit) rectangle%
      node[opacity=1,rotate=45,right=.2\unit] {#3} (\xstop,.1\unit);%
  \fi}%

%-=-=-=-=-=-=-=-=-=-=-=-=-=-=-=-=-=-=-=-=-=-=-=-=
%        BEAMER OPTIONS
%-=-=-=-=-=-=-=-=-=-=-=-=-=-=-=-=-=-=-=-=-=-=-=-=

%\setbeameroption{show notes}

%-=-=-=-=-=-=-=-=-=-=-=-=-=-=-=-=-=-=-=-=-=-=-=-=
%
%   PRESENTATION INFORMATION
%
%-=-=-=-=-=-=-=-=-=-=-=-=-=-=-=-=-=-=-=-=-=-=-=-=

\title{\textup{TRAVELING THIEF PROBLEM}}
\subtitle{}
%\date{\small{\jobname}}
\date{}
\author{}
\institute{}



\begin{document}

%-=-=-=-=-=-=-=-=-=-=-=-=-=-=-=-=-=-=-=-=-=-=-=-=
%
%   TITLE PAGE
%
%-=-=-=-=-=-=-=-=-=-=-=-=-=-=-=-=-=-=-=-=-=-=-=-=

\maketitle

%\begin{frame}[plain]
%   \titlepage
%\end{frame}

%-=-=-=-=-=-=-=-=-=-=-=-=-=-=-=-=-=-=-=-=-=-=-=-=
%
%   TABLE OF CONTENTS: OVERVIEW
%
%-=-=-=-=-=-=-=-=-=-=-=-=-=-=-=-=-=-=-=-=-=-=-=-=

%-=-=-=-=-=-=-=-=-=-=-=-=-=-=-=-=-=-=-=-=-=-=-=-=
%
%   SECTION: BACKGROUND
%
%-=-=-=-=-=-=-=-=-=-=-=-=-=-=-=-=-=-=-=-=-=-=-=-=

\section{\textup{SPLITTING THE PROBLEM: TSP SUBPROBLEM}}
%-=-=-=-=-=-=-=-=-=-=-=-=-=-=-=-=-=-=-=-=-=-=-=-=
%   FRAME:
%-=-=-=-=-=-=-=-=-=-=-=-=-=-=-=-=-=-=-=-=-=-=-=-=
\begin{frame}[c]{Trivial tour}
\vspace{-2em}
\begin{itemize}
	\item Return the trivial tour $1, 2, ..., n$ where $n$ is the number of vertices.
\end{itemize}
\vspace{2em}
\end{frame}

%-=-=-=-=-=-=-=-=-=-=-=-=-=-=-=-=-=-=-=-=-=-=-=-=
%   FRAME:
%-=-=-=-=-=-=-=-=-=-=-=-=-=-=-=-=-=-=-=-=-=-=-=-=
\begin{frame}[c]{Nearest neighbor}
\vspace{-2em}
\begin{itemize}
	\item For every city, take the nearest neighbor that has not been visited and adds it to the tour
	\item Since we have a complete graph, this always leads to a solution
	\item Usually gives good solutions for the TSP.
\end{itemize}
\vspace{2em}
\end{frame}
%-=-=-=-=-=-=-=-=-=-=-=-=-=-=-=-=-=-=-=-=-=-=-=-=
%   FRAME:
%-=-=-=-=-=-=-=-=-=-=-=-=-=-=-=-=-=-=-=-=-=-=-=-=

\begin{frame}[c]{2-opt solution}
\vspace{-2em}

	\begin{itemize}
	\item Works only for metric graphs (as our instances).
	\item Works only for metric graphs (as our instances).
	\item Double every edge and generate a Eulerian path.
	\item Whenever we visit a city that has already been visited, shortcut the path (triangular inequality).
	\end{itemize}
	\vspace{2em}
\end{frame}


%-=-=-=-=-=-=-=-=-=-=-=-=-=-=-=-=-=-=-=-=-=-=-=-=
%
%   SECTION: BACKGROUND
%
%-=-=-=-=-=-=-=-=-=-=-=-=-=-=-=-=-=-=-=-=-=-=-=-=

\section{\textup{SPLITTING THE PROBLEM: PACKING PLAN}}
%-=-=-=-=-=-=-=-=-=-=-=-=-=-=-=-=-=-=-=-=-=-=-=-=
%   FRAME:
%-=-=-=-=-=-=-=-=-=-=-=-=-=-=-=-=-=-=-=-=-=-=-=-=

\begin{frame}[c]{Constructive heuristic}
\vspace{-2em}

\begin{itemize}
	\item Generate scores $(score, u)$ for each item based on the potential profit.
	\item Choose the best possible items for this metric that don't overflow the maximal weight capacity.
\end{itemize}
\vspace{2em}
\end{frame}

%-=-=-=-=-=-=-=-=-=-=-=-=-=-=-=-=-=-=-=-=-=-=-=-=
%   FRAME: What is Beamer?
%-=-=-=-=-=-=-=-=-=-=-=-=-=-=-=-=-=-=-=-=-=-=-=-=

\begin{frame}[c]{Random Local Search}
\vspace{-2em}
\begin{block}{Neighborhood}
Two packing plans are neighbors if only if the packing status of only one item is different
\end{block}

\begin{itemize}
\item Start with the empty backpack.
\item Invert the packing status for a random item (take a neighbor of the current solution).
\item If this represents an improvement and does not exceed the weight of the knapsack, take this as the solution.
\item Repeat the process until no improvement is found for $N$ interactions.
\end{itemize}

\end{frame}
%-=-=-=-=-=-=-=-=-=-=-=-=-=-=-=-=-=-=-=-=-=-=-=-=
%   FRAME: What is Beamer?
%-=-=-=-=-=-=-=-=-=-=-=-=-=-=-=-=-=-=-=-=-=-=-=-=

\begin{frame}[c]{Evolutionary algorithm}
\vspace{-2em}

\begin{itemize}
\item Start with an empty backpack.
\item Invert the packing status of each item with probability $1/m$ ($m$ is the quantity of items).
\item If this represents an improvement and does not exceed the weight of the knapsack, take this as the solution.
\item Repeat the process until no improvement is found for $N$ interactions
\end{itemize}
\end{frame}
%-=-=-=-=-=-=-=-=-=-=-=-=-=-=-=-=-=-=-=-=-=-=-=-=
%   FRAME: What is Beamer?
%-=-=-=-=-=-=-=-=-=-=-=-=-=-=-=-=-=-=-=-=-=-=-=-=

\begin{frame}[c]{Metropolis algorithm}
\vspace{-2em}
\begin{itemize}
\item Execute a random local search but may take packing plans worst than the current solution.
\item Return the best solution seen during the whole execution of the algorithm.
\end{itemize}
\end{frame}

%-=-=-=-=-=-=-=-=-=-=-=-=-=-=-=-=-=-=-=-=-=-=-=-=
%
%   SECTION: Modèles ÉconomiquesUPDATESS
%
%-=-=-=-=-=-=-=-=-=-=-=-=-=-=-=-=-=-=-=-=-=-=-=-=

\section{\textup{RESULTS}}

%-=-=-=-=-=-=-=-=-=-=-=-=-=-=-=-=-=-=-=-=-=-=-=-=
%   FRAME:
%-=-=-=-=-=-=-=-=-=-=-=-=-=-=-=-=-=-=-=-=-=-=-=-=
\begin{frame}{Results: Uncorrelated items}
\vspace{-2em}
\begin{table}[]
	\caption{Objective value of different heuristics for a given set of instances with uncorrelated
	items. The TSP subproblem was solved using the nearest neighbor heuristic}
	\begin{tabular}[]{lrrrr}
		\toprule
		\textbf{}			& \multicolumn{1}{c}{\textbf{a280}}
		                    & \multicolumn{1}{c}{\textbf{d1655}}
		                    & \multicolumn{1}{c}{\textbf{eil51}}
		                    & \multicolumn{1}{c}{\textbf{ul1060}} \\
		\midrule                                           
		SH				& -4595 & -136782 & -6757 & -167166	\\[0.25em]
		RLS				& -2444 & -14928 & -1858 & -325761  \\[0.25em]
		EA				& 42965 & 690949 & 1843 & 300736   	\\[0.25em]
		Metropolis		& 2168 & -45453 & 1317 & -317541	\\
		\bottomrule
	\end{tabular}
	\label{tab:WindowFunctions}
\end{table}
\end{frame}


\begin{frame}{Results: Uncorrelated	items with similar weights}
\vspace{-2em}
\begin{table}[]
	\caption{Objective value of different heuristics for a given set of instances with uncorrelated
	items but with similar weights. The TSP subproblem was solved using the nearest neighbor
	heuristic}
	\begin{tabular}[]{lrrrr}
		\toprule
		\textbf{}			& \multicolumn{1}{c}{\textbf{a280}}
		                    & \multicolumn{1}{c}{\textbf{d1655}}
		                    & \multicolumn{1}{c}{\textbf{eil51}}
		                    & \multicolumn{1}{c}{\textbf{ul1060}} \\
		\midrule                                           
		SH				& -161393 & -2763579 & -39770 & -1446970	\\[0.25em]
		RLS				& 598     & -218440  & -1354  & -115155	    \\[0.25em]
		EA				& 78810   & 580623   & 746    & 446964   	\\[0.25em]
		Metropolis		& 31771   & -39170   & -1255  & -22768	\\
		\bottomrule
	\end{tabular}
	\label{tab:WindowFunctions}
\end{table}
\end{frame}


\begin{frame}{Results: Bounded strongly correlated items}
\vspace{-2em}
\begin{table}[]
	\caption{Objective value of different heuristics for a given set of instances with Bounded strongly correlated
	items. The TSP subproblem was solved using the nearest neighbor	heuristic}
	\begin{tabular}[]{lrrrr}
		\toprule
		\textbf{}			& \multicolumn{1}{c}{\textbf{a280}}
		                    & \multicolumn{1}{c}{\textbf{d1655}}
		                    & \multicolumn{1}{c}{\textbf{eil51}}
		                    & \multicolumn{1}{c}{\textbf{ul1060}} \\
		\midrule                                           
		SH				& -256694 & -1963700 & -48846 & -2241064	\\[0.25em]
		RLS				& 31394 & -29619 & -3313 & -329533	    \\[0.25em]
		EA				& 73465	& 1212683 & -1922 & 408846   	\\[0.25em]
		Metropolis		& 22808	& -24877 & -8312 & -137528	\\
		\bottomrule
	\end{tabular}
	\label{tab:WindowFunctions}
\end{table}
\end{frame}

\begin{frame}{Analyse}
\vspace{-2em}
\begin{itemize}
\item Constructive heuristic takes a lot of items increasing the time to finish the tour. This affects negatively its objective value.
\item RLS heuristic gives a reasonably good solution in a short period of time.
\item EA heuristic gives a good solution but takes more time to do it.
\item Metropolis should be better than the others iterative heuristics since we allow it to get out of local maximum. However, the problem is how choose a good value of $T$.
\end{itemize}
\end{frame}

\section{\textup{Possible improvements}}

\begin{frame}{Possible improvements}
\vspace{-2em}
\begin{itemize}
\item Start the randomized algorithms with a better departing point, maybe, our constructive heuristic.
\item Use more powerful computers which would allow more iterations for the randomized algorithms.
\item Study for an optimal "temperature" in the metropolis algorithm.
\item Try to find a direct algorithm for solving both problems at once.
\end{itemize}
\end{frame}


\end{document}